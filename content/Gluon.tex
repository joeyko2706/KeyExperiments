\chapter{The Discovery of the Gluon}

In order to organize the vast amount of particles that have been discovered up to 1961 Murray Gell-Mann and Yuval Ne'eman proposed independently from each other the Eightfold Way.
The Eightfold Way organizes hadrons in dependence on the quantum variables strangeness and isospin. 
Mesons with a spin parity configuration of $0^-$ and baryons with a spin parity configuration of $\sfrac 12^+$ can both be organized in an octet.
Baryons with a spin parity configuration can be organized in a decuplet of which the tenth particle, the $\Omega^-(sss)$ has been proposed theoretically before it has been experimentally found later \cite{Fritzsch2018}.
The early quark model was proposed by Gell-Mann and George Zweig in 1964 which consisted of only the up-, down- and strange-quark.
An electron-proton scattering event from 1968 revealed partons that took up half of the carried momentum and were thus the first indication of gluons \cite{Venker}.
A theoretical description is given by Quantum Chromodynamics (QCD) after which the baryon wave function must be antisymmetric thus leading to the introduction of colour as another quantum variable that can be either red, green or blue.
QCD is a $\text{SU}(3)$ gauge symmetry theory that describes the strong interaction.
It predicts eight gluons as a gauge boson that can interact with itself.
Particles with a colour charge can never be detected as a single particle, because the energy to separate two particles increases until a particle-antiparticle pair is produced
Due to confinement, quarks and gluons combine with other colour-carrying particles forming new hadrons in a process called hadronisation.
These hadrons can in turn form new hadrons themselves building a cascade of particles called a jet.
The first evidence for jets was observed at the Stanford Positron Electron Asymmetric Rings (SPEAR) at $\SI{7.4}{\giga\eV}$ in 1975.